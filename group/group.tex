\documentclass[compress]{beamer}

\usepackage[T1]{fontenc} 
\usepackage{amsmath}
\usepackage{array}
\usepackage{color}
\usepackage{graphicx}
\usepackage[section]{placeins} % force � mettre l'image o� on veut
\usepackage{float} %utiliser H pour forcer � mettre l'image o� on veut
\usepackage{lscape} %utilisation du mode paysage
\usepackage{pslatex}
\usepackage{multimedia}

\usetheme{Frankfurt}

\setbeamertemplate{footline}{
\leavevmode%
\hbox{\hspace*{-0.06cm}
\begin{beamercolorbox}[wd=.3\paperwidth,ht=2.25ex,dp=1ex,center]{author in head/foot}%
	\usebeamerfont{author in head/foot}\insertshortauthor%~~(\insertshortinstitute)
\end{beamercolorbox}%
\begin{beamercolorbox}[wd=.5\paperwidth,ht=2.25ex,dp=1ex,center]{section in head/foot}%
	\usebeamerfont{section in head/foot}\insertshorttitle
\end{beamercolorbox}%
\begin{beamercolorbox}[wd=.2\paperwidth,ht=2.25ex,dp=1ex,right]{section in head/foot}%
	\usebeamerfont{section in head/foot}\insertshortdate{}\hspace*{2em}
	\insertframenumber{} / \inserttotalframenumber\hspace*{2ex}
\end{beamercolorbox}}%
\vskip0pt%
}

\newcommand\bn{\boldsymbol{\nabla}}
\newcommand\bo{\boldsymbol{\Omega}}
\newcommand\br{\mathbf{r}}
\newcommand\la{\left\langle}
\newcommand\ra{\right\rangle}
\newcommand\bs{\boldsymbol}
\newcommand\red{\textcolor{red}}

\renewcommand{\(}{\left(}
\renewcommand{\)}{\right)}
\renewcommand{\[}{\left[}
\renewcommand{\]}{\right]}
\beamertemplatetransparentcovered

\title{Presentation}
\author{Damien Lebrun-Grandie \& Bruno Turcksin}\institute{Texas A\&M University, Dept. of Nuclear Engineering}
\date{}

\begin{document}

\begin{frame}
\maketitle
\end{frame}
%---------------------------------------------------------------------------------------------
\logo{\includegraphics[height=0.5cm]{../logo_Texas.jpg}}
\begin{frame}
\frametitle{Outline}
\tableofcontents[hideallsubsections]
\end{frame}
%---------------------------------------------------------------------------------------------
\section{Introduction}
\subsection{Introduction}
\begin{frame}
\frametitle{Introduction}
The projects in the computational method development group (M. Adams, R. McClarren, J. Morel and J. Ragusa) :
\begin{itemize}
\item Massively parallel transport calculation (PDT code).
\item Nuclear reactor analysis and design and reactor core and fuel assembly modeling and optimization.
\item Radiative transfer of x-rays and radiation hydrodynamics for inertial confinement fusion and high temperature shocks.
\item Multiphysics (uncertainty quantification and time- and -mesh adaptive high-order coupling techniques).
\item Spherical harmonics and other moment-based methods for particle transport.
\item Deterministic method for charged particles transport.
\item Monte-Carlo methods and hybrid deterministic/Monte-Carlo methods.
\item Neutron and gamma inverse problems and computed tomography applied to homeland security needs.
\end{itemize}
\end{frame}
%---------------------------------------------------------------------------------------------
\section{Main projects}
\subsection{PDT}
\begin{frame}
\frametitle{PDT code}
\begin{itemize}
\item PDT = Parallel Deterministic Transport
\item Main project of the group in collaboration with the computer science department (STAPL library).
\item Massively parallel code ($\sim O\(10^3\)$ to $\sim \(10^4\)$ processors).
\item Founded by the DOE for the CRASH project (Center for Radiative Shock Hydrodynamics): the goal is to understand radiative shocks and how well we can predict their behavior (uncertainty quantification). Used for the radiative transport part of the project.
\end{itemize}
\end{frame}
%---------------------------------------------------------------------------------------------
\subsection{Multiphysics}
\begin{frame}
\frametitle{Multiphysics} 
\begin{itemize}
\item High order methods for the time- and space-discretization of tightly coupled physics (use of JFNK).
\item Adaptive hp-refinement.
\item Adjoint non-linear sensitivity analysis.
\item Uncertainty quantification.
\end{itemize}
\end{frame}
%---------------------------------------------------------------------------------------------
\subsection{Charged particles}
\begin{frame}
\frametitle{Charged particles}
\begin{itemize}
\item Application for radiotherapy.
\item Coupled photon-electron transport.
\item Use the Boltzmann-CSD equation.
\item Acceleration scheme for highly anisotropic scattering.
\item Finite elements in space and energy.
\item Optimization of the dose.
\end{itemize}
\end{frame}
%---------------------------------------------------------------------------------------------
\subsection{Inverse problem}
\begin{frame}
\frametitle{Inverse problem}
\begin{itemize}
\item Application for the homeland security department.
\item Detection of radioactive material in large containers.
\item Passive and active detectors.
\end{itemize}
\end{frame}

%\frametitle{Introduction}
%\begin{frame}
%\begin{itemize}
%\item Computational Transport Theory.
%\item Efficiently Massively Parallel Implementation of Modern Deterministic Transport Calculations.
%\item Nuclear Reactor Analysis and Design.
%\item Numerical methods for radiative transfer of x-rays.
%\item Spherical harmonics and other moment-based methods for particle transport.
%\item Computational radiation hydrodynamics for inertial confinement fusion and high temperature shocks
%\item Multiphysics simulation of nuclear reactors.
%\item Uncertainty quantification for multiphysics simulations.
%\item Discretization techniques and associated multi-level solution techniques for neutral-particle and charged-particle diffusion and transport on unstructured meshes and structured meshes with adaptive refinement.
%\item Monte Carlo methods and hybrid deterministic/Monte Carlo methods.
%\item Discretization and solution techniques for multiphysics/multiscale calculations, e.g. radiation-hydrodynamics calculations.
%\item High performance computing applied to nuclear engineering and science (guaranteed solutions, goal-oriented computations, parallel techniques, and automated mesh refinement applied to multi-group neutron diffusion and transport equations).
%\item Time-adaptive, high-order accurate multi-physics coupling techniques applied to reactor safety and transient analyses. 
%\item Reactor core and fuel assembly modeling and optimization for increased performance and geological repository optimization. 
%\item Neutron and gamma inverse problems and computed tomography applied to homeland security needs.
%\end{itemize}
%\end{frame}

\end{document}
